\documentclass[a4paper]{llncs}

\usepackage[lutzsyntax]{virginialake}
\usepackage[urw-garamond]{mathdesign} % Comment this to use Times fonts
  \expandafter\ifx\csname leqslant\endcsname\relax\usepackage{txfonts}\fi
\usepackage[pdfborder={0 0 0}]{hyperref}
%\usepackage[dvipsnames,usenames]{color}
%\usepackage{draftwatermark}
\usepackage{amsmath}

\hyphenation{co-weak-en-ing}
\hyphenation{La-mar-che}
\hyphenation{quasi-poly-no-mial}

\renewcommand{\le}{\leqslant}
\renewcommand{\ge}{\geqslant}


\begin{document}

\title{Implication in Classical Logic}

\author{Tom Gundersen}

\institute{University of Bath, Bath BA2 7AY, UK}

\maketitle


\newcommand{\inte}{\mathsf{i}}
\newcommand{\wea  }{\mathsf{w}}
\newcommand{\con  }{\mathsf{c}}
\newcommand{\wead }{{\wea{\downarrow}}}
\newcommand{\cond }{{\con{\downarrow}}}
\newcommand{\weau }{{\wea{\uparrow}}}
\newcommand{\conu }{{\con{\uparrow}}}
\newcommand{\swi  }{\mathsf{s}}
\newcommand{\swio }{\mathsf{s_1}}
\newcommand{\swit }{\mathsf{s_2}}
\newcommand{\asso }{\mathsf{as}}

\[
\vlinf{\swi }{}{\vls[(\alpha.\beta).\gamma]}{\vls(\alpha.[\beta.\gamma])}\qquad
\vlinf{\swio}{}{\vls(\beta\vlim(\alpha.\gamma))}{\vls(\alpha.(\beta\vlim\gamma))}\qquad
\vlinf{\swit}{}{\vls((\alpha\vlim\beta)\vlim\gamma)}{\vls(\alpha.(\beta\vlim\gamma))}\qquad
\vlinf{\asso}{}{\vls[(\alpha\vlim\beta).\gamma]}{\vls(\alpha\vlim[\beta.\gamma])}
\]

\[
\vlinf{\cond}{}{\alpha}{\vls[\alpha.\alpha]}\qquad
\vlinf{\conu}{}{\vls(\alpha.\alpha)}{\alpha}\qquad
\vlinf{\wead}{}{\alpha}{\bot}\qquad
\vlinf{\weau}{}{\top}{\alpha}\qquad
\vlinf{\inte}{}{\vls(\alpha\vlim\alpha)}{\top}
\]

We compose derivations by concatenation and by logical connectives; if $\vlder{\Phi}{}{\beta}{\alpha}$ and $\vlder{\Psi}{}{\delta}{\gamma}$ are derivations then so are
$\vlder{\vls(\Phi.\Psi)}{}{\vls(\beta.\delta)}{\vls(\alpha.\gamma)}$, $\vlder{\vls[\Phi.\Psi]}{}{\vls[\beta.\delta]}{\vls[\alpha.\gamma]}$ and $\vlder{\vls(\Phi\vlim\Psi)}{}{\vls(\alpha\vlim\delta)}{\vls(\beta\vlim\gamma)}$. Note how derivations (\emph{i.e.} inference rules) are flipped upside down in a negative context.

It can be seen that the system is implicationally complete by translating in the obvious way from the two-sided sequent calculus. Each rule is a valid implication in classical logic.

Very vague: For the following conjecture to work we need to track the atomic flow of $\bot$, we don't care about $\top$ (I think). We need to (sometimes) create $\bot$'s using interactions and cuts (I think).

We say that $\alpha$ depends on $\beta$ in $\Phi$ if they are subformulae of the same formula and there is a directed path in the atomic flow associated with $\Phi$ from (an atom occurrence in) $\alpha$ to (an atom occurrence in) $\beta$.

We conjecture that if there is a derivation $\Phi$ from $\alpha'$ to $\beta'$ in the above system such that for every instance of the rule `$\asso$' $\beta$ depends on $\alpha$ and $\gamma$ does not, then $\alpha'\vlim\beta'$ is intuitionistically valid.

\end{document}