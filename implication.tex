\documentclass[a4paper]{article}

\usepackage[lutzsyntax]{virginialake}
%\usepackage[urw-garamond]{mathdesign} % Comment this to use Times fonts
%  \expandafter\ifx\csname leqslant\endcsname\relax\usepackage{txfonts}\fi
\usepackage[pdfborder={0 0 0}]{hyperref}
%\usepackage[dvipsnames,usenames]{color}
%\usepackage{draftwatermark}
\usepackage{amsmath}

\hyphenation{co-weak-en-ing}
\hyphenation{La-mar-che}
\hyphenation{quasi-poly-no-mial}

\renewcommand{\le}{\leqslant}
\renewcommand{\ge}{\geqslant}


\begin{document}

\title{Strongly Normalising Explicit Substitution for Intuitionistic and Classical Logic}

\author{Tom Gundersen and Michel Parigot}

\maketitle

A `path' through an atomic flow is the traditional `ai-path' (i.e. we dissalow the two `up' (resp., `down') edges of a (co)contraction from being adjacent), with the following restriction:

For every medial instance
\[
\vlinf{}{}
{\vls([A.C].[B.D])}
{\vls[(A.B).(C.D)]}
\]
every path that passes through both A and D or both B and C are discarded.

\begin{itemize}
\item Theorem: the `tamer' of `Breaking Paths...' preserves these kind of paths (it has been keeping me up at night that the tamer does not preserve the normal paths ever since I first noticed it back in Bath).
\item (Very strong) conjucture: These paths are preserved by $\beta$-reduction (in both $\lambda$ and $\lambda\mu$ calculus)
\item Conjecture (working exactly what I have been trying to prove with Lutz): any flow where all these paths are acyclic* corresponds to a valid classical logic proof (and a suitable restriction will give a intutionistic logic proof).
\end{itemize}
*) we do as Retore and we connecte edges that are in disjunctions (resp., conjuncions) in the premiss (resp., conclusion).

\[
\vlinf{}{}{\top}{A}\qquad
\vlinf{}{}{\vls(A.A)}{A}\qquad
\vlinf{}{}{\vls([A\vlim B].[A\vlim C])}{\vls[A\vlim(B.C)]}\qquad
\vlinf{}{}{B}{\vls(\vlder{}{}{A}{\Gamma}\;\;.\;\;\vlder{}{}{A\vlim B}{\Delta})}\qquad
\vlinf{}{}{A\;\;\vlim\;\;\vlder{}{}{B}{\vls(A.\Gamma)}}{\Gamma}\qquad
\]

\[
\vlinf{}{}{\vls(A\;.\;\vlinf{}{}{\top}{A})}{A}
\quad\rightarrow\quad
A\qquad
\vlinf{}{}{\vls([A\vlim B]\;.\;\vlinf{}{}{\top}{[A\vlim C]})}{\vls[A\vlim(B.C)]}
\quad\rightarrow\quad
\vls[A\vlim(B\;.\;\vlinf{}{}{\top}{C})]\qquad
\]
\[
\vlinf{}{}{\vls([A\vlim B]\;.\;[A\vlim \vlinf{}{}{\top}{C}])}{\vls[A\vlim(B.C)]}
\quad\rightarrow\quad
\vlinf{}{}{\vls([A\vlim B]\;.\;[A\vlim \top])}{\vls[A\vlim(B.\vlinf{}{}{\top}{C})]}
\]
\[
\vlderivation
{
  \vlin{}{}{\top}
  {
    \vlin{}{}{\vls(A.A)}{\vlhy{A}}
  }
}\quad\rightarrow\quad
\vlinf{}{}{\top}{A}\qquad
\vlderivation
{
  \vlin{}{}{\top}
  {
    \vlin{}{}{\vls([A\vlim B].[A\vlim C])}{\vlhy{\vls[A\vlim(B.C)]}}
  }
}\quad\rightarrow\quad
\vlinf{}{}{\top}{\vls[A\vlim(B.C)]}\qquad
\]
\[
\vlderivation
{
  \vlin{}{}{\top}
  {
    \vlin{}{}{B}{\vlhy{\vls(\vlder{}{}{A}{\Gamma}\;\;.\;\;\vlder{}{}{A\vlim B}{\Delta})}}
  }
}\quad\rightarrow\quad
\vls(\vlderivation{\vlin{}{}{\top}{\vlde{}{}{A}{\vlhy{\Gamma}}}}\;\;.\;\;\vlderivation{\vlin{}{}{\top}{\vlde{}{}{A\vlim B}{\vlhy{\Delta}}}})
\qquad
\]
\[
\vlderivation
{
  \vlin{}{}{\top}
  {
    \vlin{}{}{A\;\;\vlim\;\;\vlder{}{}{B}{\vls(A.\Gamma)}}{\vlhy\Gamma}
  }
}\quad\rightarrow^\star\quad
\vlderivation
{
  \vlin{}{}{\top}
  {
    \vlin{}{}{A\;\;\vlim\;\;\vlderivation{\vlin{}{}{\top}{\vlde{}{}{B}{\vlhy{\vls(A.\Gamma)}}}}}{\vlhy\Gamma}
  }
}\qquad
\vlderivation
{
  \vlin{}{}{\top}
  {
    \vlin{}{}{A\;\;\vlim\;\;\vlinf{}{}{\top}{\vls(A.\Gamma)}}{\vlhy\Gamma}
  }
}\quad\rightarrow\quad
\vlinf{}{}{\top}{\Gamma}\qquad
\]

$\star$) $B\neq\top$.

\end{document}